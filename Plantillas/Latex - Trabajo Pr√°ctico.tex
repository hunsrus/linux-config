\documentclass[a4paper, 12pt]{article}
\usepackage[spanish]{babel}
\usepackage[utf8]{inputenc}
\usepackage{amsmath}
\usepackage{indentfirst}
\usepackage{graphicx}
\usepackage[colorinlistoftodos]{todonotes}
\usepackage{esint}
\usepackage{multicol}

\setlength{\marginparwidth}{2cm}
\begin{document}
\begin{titlepage}
	\begin{center}
		{\large{UNIVERSIDAD TECNOLÓGICA NACIONAL}}
	\end{center}
	\vspace{15pt}
	\begin{figure}[!ht]
		\centering
		\begin{center}
			\includegraphics[width=5cm]{utn.png}
		\end{center}
	\end{figure}
	\vspace{5pt}
	\begin{center}
		{\large{FACULTAD REGIONAL PARANÁ}}
		\vspace{5pt}
		\begin{center}
			\vspace{15pt}
			\normalsize{CARRERA: Ingeniería Electrónica\\
						CÁTEDRA: Nombre de la Cátedra\\}
			\vspace{50pt}
			\huge\bfseries{Trabajo Práctico N°X\\
						Tema del trabajo práctico\\}
			\vspace{50pt}
		\end{center}
		
		\begin{flushleft}
			\begin{center}
				ALUMNOS:\\
				Battaglia Carlo\\
				Efchi Mauro\\
				Escobar Gabriel\\
			\end{center}
		\end{flushleft}
		
		\begin{center}
			\vspace{\fill}
			\normalsize{Paraná,}
			\today
		\end{center}
	\end{center}
\end{titlepage}

\newpage
\pagenumbering{gobble}
\tableofcontents

\newpage
\pagenumbering{arabic}
\numberwithin{equation}{section}

\section{Enunciado 1}
Resolvente
\begin{equation}
	x=\frac{-b\pm\sqrt{b-4ac}}{2a}
\end{equation}
\subsection{Consigna A}
Ecuaciones de Maxwell en forma diferencial
\begin{equation}
	\nabla\times B=\mu_{0} J+\mu_{0}\epsilon_{0}\frac{\partial E}{\partial t}
\end{equation}
\begin{equation}
	\nabla\times E=-\frac{\partial B}{\partial t}
\end{equation}
\begin{equation}
	\nabla\cdot E=\frac{\rho}{\epsilon_{0}}
\end{equation}
\begin{equation}
	\nabla\cdot B=0
\end{equation}
\subsection{Consigna B}
Ecuaciones de Maxwell en forma integral
\begin{equation}
	\oint_{\partial\Sigma}{B\cdot\partial l}=\mu_{0}\Big(\iint_{\Sigma}{J\cdot dS}+\epsilon_{0}\frac{d}{dt}\iint_{\Sigma}{E\cdot dS}\Big)
\end{equation}
\begin{equation}
	\oint_{\partial\Sigma}{E\cdot\partial l}=-\frac{d}{dt}\iint_{\Sigma}{B\cdot dS}
\end{equation}
\begin{equation}
	\oiint_{\partial\Omega}{E\cdot dS}=\frac{1}{\epsilon_{0}}\iiint_{\Omega}{\rho dV}
\end{equation}
\begin{equation}
	\oiint_{\partial\Omega}{B\cdot dS}=0
\end{equation}
\section{Enunciado 2}
\begin{equation}
	E=mc^2
\end{equation}
In the equation, the increased relativistic mass $m$ of a body times the speed of light squared $c^2$ is equal to the kinetic energy $E$ of that body.
\newpage
\subsection{Consigna A}
\begin{multicols}{2}
De todas tus mentiras guardo un mal sabor
Y mentiras ya no quiero más
No pierdas más tu tiempo en pedir perdón
Pues te juro que no me vuelvo atrás

No me pidas que haga la locura
De creerte una vez más
No más regálame la última luna
Una noche que no olvide jamás

Desnúdame de a poco
Y bésame a lo loco
Invéntame un momento
Que no tenga final

Aprisióname en tus brazos
Quiébrame en pedazos
Arrójalos al viento
Ámame una vez más

No hay nada mas difícil que decirte adiós
Porque sé muy bien que nunca más
Podré olvidar la música que hay en tu voz
El perfume de tu piel, tu mirar
%\columnbreak
Sé que me esperan horas muy oscuras
Y sé que voy a llorar
Pero hoy regálame la última Luna
Y una noche que no olvide jamás

Enrédate en mi pelo
Consúmeme en tu fuego
Muérdeme los labios
No me tengas piedad

Devórame esta noche
Con besos que me asombren
Y que mi propio nombre
Me hagan olvidar

Desnúdame de a poco
Y bésame a lo loco
Invéntame un momento
Que no tenga final

Aprisióname en tus brazos
Y quiébrame en pedazos
Arrójalos al viento
Ámame una vez más
\end{multicols}
\section{Enunciado 3}
Desarrollo consigna 3

\newpage
\section{Referencias}
 
[1] Shanmugan, K.S. -- Digital and Analog Communication Systems, 1° Edición;

[2] Pueyo, H.O. -- Marco, C. -- Análisis de Modelos Circuitales, 1° Edición;

\end{document}
